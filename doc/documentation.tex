% Author: Radoslav Grenčík <xgrenc00@stud.fit.vutbr.cz>
% Author: Róbert Hubinák <xhubin03@stud.fit.vutbr.cz>


\documentclass[a4paper, 11pt]{article}


\usepackage[czech]{babel}
\usepackage[utf8]{inputenc}
\usepackage[left=2cm, top=3cm, text={17cm, 24cm}]{geometry}
\usepackage{times}
\usepackage{graphicx}
\usepackage[hyphens]{url}
\usepackage[unicode, colorlinks, hypertexnames=false, citecolor=blue]{hyperref}
\usepackage[czech, boxed]{algorithm2e}
\usepackage[perpage]{footmisc}
\usepackage{hyperref}
\usepackage{multirow}
\usepackage[normalem]{ulem}
\useunder{\uline}{\ul}{}


\begin{document}
	%##########################################################################%
	% TITLE PAGE
	%##########################################################################%
	\begin{titlepage}
		\begin{center}
			\includegraphics[width=0.77 \linewidth]{FIT_logo.pdf}

			\vspace{\stretch{0.382}}

			\Huge{Simulačná štúdia} \\
			\LARGE{\textbf{Varianta 9: Plasty}} \\

			\vspace{\stretch{0.618}}
		\end{center}

		\begin{minipage}{0.5 \textwidth}
			\Large
			\today
		\end{minipage}
		\hfill
		\begin{minipage}[r]{0.5 \textwidth}
			\Large
			\begin{tabular}{ll}
				Radoslav Grenčík & (xgrenc00) \\
				Róbert Hubinák & (xhubin03)
			\end{tabular}
		\end{minipage}
	\end{titlepage}



	%##########################################################################%
	% TABLE OF CONTENTS
	%##########################################################################%
	\clearpage
	\thispagestyle{empty}
	\tableofcontents



	%##########################################################################%
	%##########################################################################%
	\clearpage
	\pagenumbering{arabic}
	\setcounter{page}{1}

	\section{Úvod}

	V tejto práci sa rozoberá problém plastov na našej planéte. Cieľom práce je
	vytvoriť model, ktorý popisuje kritickú situáciu s prebytkom plastového odpadu
	na našej planéte. V práci sa rozoberá hlavne problém s jednoúčelovými a jednorázovými
	plastovými výrobkami ako sú rôzne obaly poprípade iné jednorázové výrobky. Tieto
	výrobky tvoria najväčšiu časť plastového odpadu. V práci sa vyskytujú rôzne
	experimenty, ktorých zmyslom je demonštrovať, čo sa stane ak okamžite neznížime
	produkciu plastového odpadu, ako na množstvo plastového odpadu vplýva recyklácia
	a iné faktory a ako dlho by trvalo zbaviť sa všetkého plastového odpadu aj keby
	sa okamžite prestalo s produkciou akýchkoľvek plastových výrobkov.

	\subsection{Autori, zdroje}

	Projekt vypracovali študenti VUT FIT v Brne Radoslav Grenčík a Róbert Hubinák.

	K vypracovaniu projektu boli využité poznatky a študijné texty z predmetu
	Modelování a simulace, ktorý sa vyučuje na VUT FIT v Brne. Ako zdroj údajov
	slúžili rôzne štúdie a články na internete a takisto vlastné meranie.

	\subsection{Overovanie validity modelu}

	Validita modelu bola overovaná experimentovaním a porovnávaním výsledkov s
	reálnymi nameranými dátami, ktoré boli čerpané z overených zdrojov.

	%##########################################################################%
	%##########################################################################%
	\section{Rozbor témy a použitých metód/technológií}

	Systém modeluje životný cyklus plastu - od jeho vzniku až po rozklad.
	Podľa článku na portále \textbf{Euractiv} \cite{plastic_Europe} celosvetová produkcia
	plastu stúpa v roku 2018 bolo vyrobených 359 miliónov ton plastu, čo je 3,2\%
	nárast oproti roku 2017.

	Vyprodukovaný plast môže byť stále použitý, môže sa z neho stať odpad, môže byť
	spálený alebo zrecyklovaný. Podľa článkov na portáloch \textbf{Our World in Data} \cite{plastic_pollution_stats}
	a \textbf{ScienceAdvances} \cite{plastic_sciencemag} je približne 30\% plastu stále použitých, približne 56\%
	je odpad, približne 8\% je spálených a len približne 6\% je zrecyklovaných.
	Ďalej je v týchto článkoch spomenutý fakt, že približne 20\% zo zrecyklovaného
	odpadu sa znovu použije, takisto približne 20\% sa spáli a až 60\% zrecyklovaného
	odpadu ide na skládky.

	Podľa grafov z portálu \textbf{European Parliamentary Research Service Blog} \cite{plastic_graph}
	je väčšina plastového odpadu tvorená hlavne plastovými obalmi a druhé miesto
	tvoria rôzne plastové výrobky nespadajúce do katégorií: elektronika,
	automobilový priemysel ani stavebníctvo. Model sa preto zameriava práve na spomínaný
	druh plastového odpadu. Podľa článku na portále \textbf{EcoWatch} \cite{beach_cleanup}
	je práve top 10 nájdených vecí pri medzinárodnom čistení pláží hnutím Ocean Conservancy
	v roku 2018 plastový odpad a to hlavne cigaretové ohorky a rôzne plastové obaly
	alebo iné jednorázové produkty z plastu.

	%##########################################################################%
	%##########################################################################%
	\section{Koncepcia metódy, prístupu, modelu}
	\label{model:uvod}

	Údaj o celosvetovej produkcii plastu bol zjednodušený a v simulačnom modeli sa
	generuje každý deň 1 milión ton plastu čo je vo výsledku 365 miliónov ton
	plastu ročne. V simulačnom modeli sa dá nastaviť ročný prírastok v produkcii
	plastu. Priestupné roky zanedbávame pretože pri takomto množstve je tento
	údaj zanedbateľný. Simulačný model si sám počíta čas, za ktorý sa generuje 1
	milión ton plastu na základe ročnej produkcie plastu.

	V článku na portále \textbf{EcoWatch} \cite{beach_cleanup} sú spomenuté množstvá jednotlivých
	vyzbieraných vecí. Na základe týchto množstiev a vlastného merania - približná hmotnosť
	predmetov bola získaná vážením rôznych zástupcov určitého druhu a spriemerovaním - bola
	vypočítaná celková hmotnosť nájdených predmetov v jednotlivých kategóriách.
	Nasledovne boli predmety zoskupené do kategórií podľa doby rozkladu.
	Údaje o dobách rozkladu boli problematickým údajom, pretože sa na rôznych stránkach
	vyskytujú rôzne údaje. Údaje získané z nasledovných stránok nie sú úplne presné,
	avšak pre vytvorenie si predstavy o probléme s plastovým odpadom sú dostačujúce.
	Údaje boli získané z nasledovných stránok \cite{decomposition1}, \cite{decomposition2}, \cite{decomposition3}, \cite{decomposition4}, \cite{decomposition5}.
	Výsledky sú v tabuľke \ref{tab:1}.

	\begin{table}[]
	\begin{tabular}{llllll}
	\textbf{}                                                                          & \textbf{MNOŽSTVO} & \textbf{\begin{tabular}[c]{@{}l@{}}KUSOVÁ\\ HMOTNOSŤ\end{tabular}} & \textbf{\begin{tabular}[c]{@{}l@{}}CELKOVÁ\\ HMOTNOSŤ\end{tabular}} & \textbf{\begin{tabular}[c]{@{}l@{}}DOBA\\ ROZKLADU\end{tabular}} & \textbf{KATEGÓRIA} \\ \hline
	\textbf{\begin{tabular}[c]{@{}l@{}}cigaretový ohorok/\\ drobný odpad\end{tabular}} & 2412151           & 1,4 g                                                              & 3377 kg                                                             & 5-10 rokov                                                       & A                  \\ \hline
	\textbf{slamka}                                                                    & 643562            & 0,42 g                                                             & 270 kg                                                              & 200 rokov                                                        & B                  \\ \hline
	\textbf{PET fľaša}                                                                 & 1569135           & 30 g                                                               & 47074 kg                                                            & 450 rokov                                                        & \multirow{4}{*}{C} \\
	\textbf{PET vrchnák}                                                               & 1091107           & 2 g                                                                & 2182 kg                                                             & 450 rokov                                                        &                    \\
	\textbf{plastový vrchnák}                                                          & 624878            & 3 g                                                                & 1874 kg                                                             & 450 rokov                                                        &                    \\
	\textbf{\begin{tabular}[c]{@{}l@{}}"take away"\\ box z plastu\end{tabular}}        & 632874            & 4,5 g                                                              & 2848 kg                                                             & 450 rokov                                                        &                    \\ \hline
	\textbf{igelitová taška}                                                           & 757523            & 5,5 g                                                              & 4166 kg                                                             & 20 rokov                                                         & \multirow{3}{*}{D} \\
	\textbf{plastové vrece}                                                            & 746211            & 5,5 g                                                              & 4104 kg                                                             & 20 rokov                                                         &                    \\
	\textbf{fólia/drobný obal}                                                         & 1739743           & 2 g                                                                & 3479 kg                                                             & 20 rokov                                                         &                    \\ \hline
	\textbf{\begin{tabular}[c]{@{}l@{}}"take away"\\ box z peny\end{tabular}}          & 580570            & 4,5 g                                                              & 2612 kg                                                             & 50-80 rokov                                                      & E                  \\
	\multicolumn{3}{r}{{\ul SPOLU:}}                                                                                                                                             & \multicolumn{3}{l}{{\ul 71986 kg}}
	\end{tabular}
	\caption{Tabuľka top 10 nájdených predmetov a výsledky meraní}
	\label{tab:1}
	\end{table}

	Nakoniec bola vypočítaná percentuálna zastúpenosť jednotlivých kategórií v
	celkovej hmotnosti vyzbieraného odpadu. Kategória A má zastúpenie 5\%, kategória B 0,4\%,
	kategória C 75\%, kategória D 16\% a kategória E 3,6\%. Tieto údaje boli použité pri tvorbe
	Petriho siete \ref{appendix:petri_net}.

	\subsection{Popis konceptuálneho modelu}

	Na vstupe modelu - príloha \ref{appendix:petri_net} sa nachádza časovaný prechod
	s dĺžkou prechodu 1 deň. Za túto časovú jednotku sa na vstupe vygeneruje 1 milión
	ton plastu. Plast potom môže prejsť do 4 stavov:
	\begin{itemize}
		\item pravdepodobnosť 35\% - Plast sa znovu použije a posiela sa na vstup systému.
		\item pravdepodobnosť 56\% - Plast sa stáva odpadom a môže ďalej prejsť do 5 stavov.
		\item pravdepodobnosť 8\% - Plast sa spáli a stáva sa rozloženým - opúšťa systém.
		\item pravdepodobnosť 6\% - Plast sa pošle na recykláciu a môže ďalej prejsť do 3 stavov.
	\end{itemize}
	Pokiaľ sa plast stal odpadom prejde do jedného z nasledujúcich stavov:
	\begin{itemize}
		\item pravdepodobnosť 5\% - Stáva sa cigaretovým ohorkom poprípade iným drobným odpadom. Odpad sa stáva rozloženým za 5-10 rokov - opúsťa systém.
		\item pravdepodobnosť 0,4\% - Stáva sa slamkou. Odpad sa stáva rozloženým za exponenciálne 200 rokov - opúsťa systém.
		\item pravdepodobnosť 75\% - Stáva sa o PET fľašou/vrchnákom poprípade plastovým obalom/vrchnákom. Odpad sa stáva rozloženým za exponenciálne 450 rokov - opúsťa systém.
		\item pravdepodobnosť 16\% - Stáva sa taškou poripáde plastovým vreckom či fóliou. Odpad sa stáva rozloženým za exponenciálne 20 rokov - opúsťa systém.
		\item pravdepodobnosť 3,6\% - Stáva sa penovým "take away" obalom na jedlo. Odpad sa stáva rozloženým za exponenciálne 50 rokov - opúšťa systém.
	\end{itemize}
	Pokiaľ sa plast pošle na recykláciu prejde do jedného z nasledujúcich stavov::
	\begin{itemize}
		\item pravdepodobnosť 20\% - Plast sa spáli a stáva sa rozloženým - opúšťa systém.
		\item pravdepodobnosť 20\% - Plast sa znovu použije a posiela sa na vstup systému.
		\item pravdepodobnosť 60\% - Plast sa stáva odpadom a môže ďalej prejsť do 5 stavov.
	\end{itemize}

	\subsection{Forma konceptuálneho modelu}

	Model je vyjadrený formou Petriho siete - príloha \ref{appendix:petri_net}.

	%##########################################################################%
	%##########################################################################%
	\section{Architektúra simulačného modelu/simulátoru}
	\label{architecture:uvod}
	Hlavnými komponentami implementačnej časti projektu sú triedy \texttt{Production}
	a \texttt{Plastic}. Trieda \texttt{Production} dedí od tiedy \texttt{Event}
	a stará sa o generovanie a aktiváciu procesov ktoré spracovávame. Životný cyklus
	týchto procesov je popísaný v triede \texttt{Plastic}. Program takisto obsahuje triedu
	\texttt{ArgumentParser}, ktorá sa stará o spracovanie argumentov programu.

	\subsection{Mapovanie konceptuálneho modelu do simulačného modelu}
	Ako už bolo spomenuté v úvode kapitoly \ref{architecture:uvod} o generovanie procesov
	vstupujúcich do systému sa stará trieda \texttt{Production}. Jeden tento proces predstavuje
	jeden milión ton plastu. Po vygrenerovaní je proces rozdelený do jednej zo 4 vetiev, ktoré
	predstavujú stavy popísané v modeli \ref{appendix:petri_net} (recyklovaný,skládka...). Rozdelenie
	je vo forme intervalov, ktoré zodpovedajú percentám v modeli. O náhodnosť rozdelenia sa stará
	funkcia \texttt{Random()}. Po tom čo prejde proces do tohto stavu, inkrementuje sa celočíselná premenná
	ktorá predstavu množstvo plastu v danom stave. Ak prejde proces do stavu recyklácie je následne
	opäť náhodne rozdelení do stavov podľa rozdelenia v modeli. Procesy ktoré sa dostali do stavu skládka
	sú takisto rozdelené a podľa kategórie, do ktorej spadajú im je nastavené čakanie funkciou \texttt{Wait()}.
	Ak takýto čakajúci proces stihne skončiť pred skončením simulácie považujeme ho za rozložený.

	\subsection{Spustenie simulačného modelu, parametre programu}
	Simulačný model je nutné pred spustením preložiť príkazom make alebo make run (tento príkaz po preklade spustí program). Simulačný model je možné spustiť ako bez parametrov, tak s nimi, a to v ľubovoľnom poradí. Ak užívateľ nezadá parametre, je program spustený s prednastavenými parametrami.

	\subsubsection{Popis parametrov programu}
	\begin{itemize}
		\item \texttt{-y}\quad Počet rokov simulácie [Prednastavená hodnota: 10 rokov]
		\item \texttt{-r}\quad Percento recyklovaných platov. Maximálna percentuálna hodnota je nastavená na 63\%, pretože recyklácia sa netýka plastu ktorý je znovapoužitý a spálený. [Prednastavená hodnota: 6 (viz \ref{appendix:petri_net})]
		\item \texttt{-s}\quad Percento úspešne zrecyklovaných plastov(znovupoužitých). Maximálna hodnota je nastavená na 80\% pretože 20\% z recyklovaných plasov sa spáli [Prednastavená hodnota: 20 (viz \ref{appendix:petri_net})]
		\item \texttt{-i}\quad Percentuálny ročný nárast produkcie plastov [Prednastavená hodnota: 0]
	\end{itemize}


	%##########################################################################%
	%##########################################################################%
	\section{Podstata simulačných experimentov a ich priebeh}
	Cielom experimentov bolo overiť verejne dostupné informácie o problamatike plastového odpadu vo svete a navrhnúť vhodný a zrealizovateľný plán ako zastaviť nadmerné znečistenie našej planéty. Cieľom posledného experimentu bolo simuláciou zistiť koľko by približne trvalo rozloženie všetkého plastu ktorý sa aktuálne na planéte vyskytuje.


	%##########################################################################%
	%##########################################################################%
	\section{Zhrnutie simulačných experimentov a záver}



	%##########################################################################%
	% CITATIONS
	%##########################################################################%
	\clearpage
	\bibliographystyle{czechiso}
	\renewcommand{\refname}{Literatúra}
	\bibliography{documentation}



	%##########################################################################%
	% ADDITIONS
	%##########################################################################%
	\clearpage
	\appendix


	\section{Petriho sieť}
	\label{appendix:petri_net}

	\begin{figure}[ht]
		\centering
		\includegraphics[width=1 \linewidth]{IMSpetri.pdf}

		\caption{Petriho sieť}
	\end{figure}
\end{document}
