% Author: Radoslav Grenčík <xgrenc00@stud.fit.vutbr.cz>
% Author: Róbert Hubinák <xhubin03@stud.fit.vutbr.cz>


\documentclass[a4paper, 11pt]{article}


\usepackage[czech]{babel}
\usepackage[utf8]{inputenc}
\usepackage[left=2cm, top=3cm, text={17cm, 24cm}]{geometry}
\usepackage{times}
\usepackage{graphicx}
\usepackage[hyphens]{url}
\usepackage[unicode, colorlinks, hypertexnames=false, citecolor=blue]{hyperref}
\usepackage[czech, boxed]{algorithm2e}
\usepackage[perpage]{footmisc}


\begin{document}
	%##########################################################################%
	% TITLE PAGE
	%##########################################################################%
	\begin{titlepage}
		\begin{center}
			\includegraphics[width=0.77 \linewidth]{FIT_logo.pdf}

			\vspace{\stretch{0.382}}

			\Huge{Simulačná štúdia} \\
			\LARGE{\textbf{Plasty}} \\
			\Large{Varianta 9: Plasty}

			\vspace{\stretch{0.618}}
		\end{center}

		\begin{minipage}{0.5 \textwidth}
			\Large
			\today
		\end{minipage}
		\hfill
		\begin{minipage}[r]{0.5 \textwidth}
			\Large
			\begin{tabular}{ll}
				Radoslav Grenčík & (xgrenc00) \\
				Róbert Hubinák & (xhubin03)
			\end{tabular}
		\end{minipage}
	\end{titlepage}



	%##########################################################################%
	% TABLE OF CONTENTS
	%##########################################################################%
	\clearpage
	\pagenumbering{roman}
	\setcounter{page}{1}
	\tableofcontents



	%##########################################################################%
	%##########################################################################%
	\clearpage
	\pagenumbering{arabic}
	\setcounter{page}{1}

	\section{Úvod}

	V tejto práci sa rozoberá problém plastov na našej planéte. Cieľom práce je
	vytvoriť model, ktorý popisuje kritickú situáciu s prebytkom plastového odpadu
	na našej planéte. V práci sa rozoberá hlavne problém s jednoúčelovými a jednorázovými
	plastovými výrobkami ako sú rôzne obaly poprípade iné jednorázové výrobky. Tieto
	výrobky tvoria najväčšiu časť plastového odpadu. V práci sa vyskytujú rôzne
	experimenty, ktorých zmyslom je demonštrovať, čo sa stane ak okamžite neznížime
	produkciu plastového odpadu, ako na množstvo plastového odpadu vplýva recyklácia
	a iné faktory a ako dlho by trvalo zbaviť sa všetkého plastového odpadu aj keby
	sa okamžite prestalo s produkciou akýchkoľvek plastových výrobkov.

	\subsection{Autori, zdroje}

	Projekt vypracovali študenti VUT FIT v Brne Radoslav Grenčík a Róbert Hubinák.

	K vypracovaniu projektu boli využité poznatky a študijné texty z predmetu
	Modelování a simulace, ktorý sa vyučuje na VUT FIT v Brne. Ako zdroj údajov
	slúžili rôzne štúdie a články na internete a takisto vlastné meranie.

	%##########################################################################%
	%##########################################################################%
	\section{Rozbor témy a použitých metód/technológií}



	%##########################################################################%
	%##########################################################################%
	\section{Koncepcia metódy, prístupu, modelu}



	\subsection{Koncepcia - model}


	\subsection{Koncepcia - implementácia}



	%##########################################################################%
	%##########################################################################%
	\section{Architektúra simulačného modelu/simulátoru}



	%##########################################################################%
	%##########################################################################%
	\section{Podstata simulačných experimentov a ich priebeh}



	%##########################################################################%
	%##########################################################################%
	\section{Zhrnutie simulačných experimentov a záver}



	%##########################################################################%
	% CITATIONS
	%##########################################################################%
	\clearpage
	\bibliographystyle{czechiso}
	\renewcommand{\refname}{Literatúra}
	\bibliography{documentation}



	%##########################################################################%
	% ADDITIONS
	%##########################################################################%
	\clearpage
	\appendix


	\section{Petriho sieť}
	\label{appendix:petri_net}

	\begin{figure}[ht]
		\centering
		\includegraphics[width=1 \linewidth]{IMSpetri.pdf}

		\caption{Petriho sieť}
	\end{figure}
\end{document}
